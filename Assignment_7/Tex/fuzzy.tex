\documentclass[a4paper,11pt]{article}
\usepackage[utf8]{inputenc}
\usepackage[T1]{fontenc}
\usepackage{lmodern}
\usepackage{graphicx}
\usepackage{hyperref}
\usepackage[margin=2.5cm]{geometry}
\usepackage{amsmath,amssymb}
\usepackage{xcolor}
\usepackage{listings}
\usepackage{float}
\usepackage{booktabs}
\usepackage{enumitem}
\usepackage{babel}

% Kustom warna untuk syntax highlighting
\definecolor{codegreen}{rgb}{0,0.6,0}
\definecolor{codegray}{rgb}{0.5,0.5,0.5}
\definecolor{codepurple}{rgb}{0.58,0,0.82}
\definecolor{backcolour}{rgb}{0.96,0.96,0.96}

% Konfigurasi Python listings
\lstdefinestyle{mystyle}{
    backgroundcolor=\color{backcolour},   
    commentstyle=\color{codegreen},
    keywordstyle=\color{blue},
    numberstyle=\tiny\color{codegray},
    stringstyle=\color{codepurple},
    basicstyle=\ttfamily\footnotesize,
    breakatwhitespace=false,         
    breaklines=true,                 
    captionpos=b,                    
    keepspaces=true,                 
    numbers=left,                    
    numbersep=5pt,                  
    showspaces=false,                
    showstringspaces=false,
    showtabs=false,                  
    tabsize=2
}

\lstset{style=mystyle}

\title{Sistem Fuzzy Obstacle Avoidance PioneerP3DX}
\newcommand{\youtubelink}{\href{https://youtu.be/wl1X7vWXRO4}{Video Demo}}
\author{Figo Arzaki Maulana}
\date{28 April 2025}

\begin{document}

\maketitle

\section{Deskripsi Sistem}
\youtubelink

Sistem ini menggunakan 6 sensor ultrasonik pada robot PioneerP3DX untuk program obstacle avoidance. Data jarak dari setiap sensor difuzzifikasi menjadi dua fuzzy membership: Near dan Far. Hasil fuzzy inference menghasilkan sinyal PWM untuk motor kiri dan kanan. Pada sistem ini digunakan 2 input dan 1 output untuk setiap roda kanan dan kiri. Sistem menggunakan singleton PWM output dengan nilai -5 dan 5 namun pada velocity dilakukan normalisasi kecepatan menjadi -1 hingga 1. Hal ini dilakukan melalui tuning dan memberikan hasil yang lebih bagus daripada menggunakan singleton PWM dengan nilai -1 dan 1.

\section{Membership Function}
Proses fuzzifikasi menggunakan fungsi segitiga. Implementasi fungsi keanggotaan untuk "Near" dan "Far" terhadap jarak sensor adalah sebagai berikut:

\begin{lstlisting}[language=Python, caption=Implementasi Fungsi Keanggotaan Segitiga]
def triangle_membership(x, a, b, c, FD0, FD2):
    if x < a:
        FD = FD0
    elif x >= a and x < b:
        FD = (x - a) / (b - a)
    elif x >= b and x < c:
        FD = (c - x) / (c - b)
    elif x >= c:
        FD = FD2
    return FD

def ultrasound_membership(x):
    near = triangle_membership(x, 0.2, 0.2, 0.3, 1, 0)
    far = triangle_membership(x, 0.2, 0.3, 0.3, 0, 1)
    y = np.array([[near], [far]], dtype=float)
    return y
\end{lstlisting}

Fungsi keanggotaan menggunakan parameter-parameter berikut:
\begin{itemize}
    \item \textbf{Near}: Segitiga dengan a=0.2, b=0.2, c=0.3, FD0=1, FD2=0
    \item \textbf{Far}: Segitiga dengan a=0.2, b=0.3, c=0.3, FD0=0, FD2=1
\end{itemize}

\section{Rule Base dalam Tabel}
Basis aturan fuzzy mendefinisikan bagaimana sistem merespons input dari sensor. Terdapat tabel aturan terpisah untuk motor kiri dan kanan:

\subsection{Aturan Motor Kiri}
\begin{table}[H]
\centering
\begin{tabular}{|c|c|c|}
\hline
\multicolumn{1}{|c|}{} & \multicolumn{2}{c|}{Sensor Kiri} \\
\cline{2-3}
\multicolumn{1}{|c|}{Sensor Kanan} & Near & Far \\
\hline
Near & IF Near/Near → -5 & IF Far/Near → -5 \\
\hline
Far & IF Near/Far → +5 & IF Far/Far → +5 \\
\hline
\end{tabular}
\caption{Tabel Aturan Fuzzy untuk Motor Kiri}
\end{table}

\subsection{Aturan Motor Kanan}
\begin{table}[H]
\centering
\begin{tabular}{|c|c|c|}
\hline
\multicolumn{1}{|c|}{} & \multicolumn{2}{c|}{Sensor Kiri} \\
\cline{2-3}
\multicolumn{1}{|c|}{Sensor Kanan} & Near & Far \\
\hline
Near & IF Near/Near → -5 & IF Far/Near → +5 \\
\hline
Far & IF Near/Far → -5 & IF Far/Far → +5 \\
\hline
\end{tabular}
\caption{Tabel Aturan Fuzzy untuk Motor Kanan}
\end{table}

Dalam kode, aturan-aturan ini diimplementasikan sebagai matriks:

\begin{lstlisting}[language=Python, caption=Implementasi Tabel Aturan]
rule_table_left = np.array([[0, 0], [1, 1]], dtype=int)
rule_table_right = np.array([[0, 1], [0, 1]], dtype=int)
singleton_PWM_outputs = np.array([[-5], [5]], dtype=float)
\end{lstlisting}

Di mana indeks 0 sesuai dengan nilai -5 dan indeks 1 sesuai dengan nilai +5 dalam nilai output singleton.

\section{Defuzzifikasi dan Plot Output Crisp}
Setelah inferensi, output didefuzzifikasi dengan metode weighted average. Implementasi melibatkan perhitungan output crisp berdasarkan evaluasi aturan:

\begin{lstlisting}[language=Python, caption=Implementasi Defuzzifikasi]
defuzz_table = []
for idx in range(3):
    left_idx = idx
    right_idx = len(output_singleton) - idx - 1
    
    left_memberships = output_singleton[left_idx]
    right_memberships = output_singleton[right_idx]
    
    num_left = 0
    den_left = 0
    num_right = 0
    den_right = 0
    
    for r in range(2):  # near/far
        for l in range(2):  # near/far
            tab_idx_left = rule_table_left[r][l]
            tab_idx_right = rule_table_right[r][l]
            
            fd1andfd2 = float(min(left_memberships[l], right_memberships[r]))
            
            num_left += fd1andfd2 * singleton_PWM_outputs[tab_idx_left]
            den_left += fd1andfd2
            num_right += fd1andfd2 * singleton_PWM_outputs[tab_idx_right]
            den_right += fd1andfd2
    
    crisp_left = num_left / den_left if den_left > 0 else 0
    crisp_right = num_right / den_right if den_right > 0 else 0
    crisp_out.append([crisp_left, crisp_right])
\end{lstlisting}

\section{Konfigurasi Sensor dan Pembebanan}
Robot menggunakan 6 sensor ultrasonik dengan indeks sensor [1,2,3,4,5,6]. Dari keenam sensor tersebut, data diambil secara berpasangan mulai dari sensor terluar hingga sensor terdalam:

\begin{itemize}
    \item Sensor pasang 1 (indeks 1 \& 6) dengan bobot 1.5
    \item Sensor pasang 2 (indeks 2 \& 5) dengan bobot 2.5
    \item Sensor pasang 3 (indeks 3 \& 4) dengan bobt 3.5
\end{itemize}

Bobot ini memengaruhi perhitungan weighted crisp sebelum normalisasi ke rentang [-1, 1]. Nilai tersebut ditentukan berdasarkan urgensi deteksi obstacle:
\begin{itemize}
    \item Sensor yang berada di samping ketika mendeteksi obstacle berarti akan membuat robot bersenggolan atau hanya terserempet sehingga urgensinya tidak terlalu besar.
    \item Pasangan sensor yang ada di tengah memiliki urgensi yang sangat besar karena robot pasti akan bertabrakan jika lurus.
\end{itemize}

\begin{lstlisting}[language=Python, caption=Perhitungan Weighted Crisp Output]
weights = [1.5, 2.5, 3.5]
weighted_crisp = [0, 0]
for i in range(len(crisp_out)):
    weighted_crisp[0] += crisp_out[i][0] * weights[i]
    weighted_crisp[1] += crisp_out[i][1] * weights[i]
weighted_crisp[0] /= sum(weights)
weighted_crisp[1] /= sum(weights)

max_velocity = 4
weighted_crisp[0] = max(-1, min(1, weighted_crisp[0] / max_velocity))
weighted_crisp[1] = max(-1, min(1, weighted_crisp[1] / max_velocity))
\end{lstlisting}

\section{Visualisasi}
Sistem ini menyertakan visualisasi real-time dari fungsi keanggotaan dan pembacaan sensor:

\begin{lstlisting}[language=Python, caption=Pengaturan Visualisasi]
plt.ion()
fig, ax = plt.subplots(figsize=(10, 6))
plt.title('Distance Membership Functions')
plt.xlabel('Distance (m)')
plt.ylabel('Membership Value')
plt.grid(True)
plt.xlim([0, 1])
plt.ylim([0, 1.1])

ax.plot(dis_eval, near_vals, 'b-', label='Near')
ax.plot(dis_eval, far_vals, 'g-', label='Far')
ax.fill_between(dis_eval, near_vals, alpha=0.2, color='blue')
ax.fill_between(dis_eval, far_vals, alpha=0.2, color='green')
plt.legend()
\end{lstlisting}

\appendix
\section{Kode Lengkap}
\lstinputlisting[language=Python, caption=Implementasi Lengkap Fuzzy Obstacle Avoidance]{fuzzy.py}

\end{document}